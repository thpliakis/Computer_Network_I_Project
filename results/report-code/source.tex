\documentclass[10pt,a4paper]{article}
\usepackage{hyperref}
\usepackage[a4paper, total={6in, 10in}]{geometry}
\usepackage{fancyhdr}
\pagestyle{fancy}
\usepackage[greek,english]{babel}

\title{ 
    \begin{otherlanguage}{greek}
    Δίκτυα Υπολογιστών Ι
    \end{otherlanguage}
    }
\author{
    \begin{otherlanguage}{greek}
    Θωμάς Πλιάκης
    \end{otherlanguage}
    \\ 
    tpliakis@ece.auth.gr 
    \\
    AEM: 9018
    }
    
\lhead{Computer Networks I - Thomas Pliakis - 9018}


\date{\today}
\usepackage{listings}
\usepackage{color}

\definecolor{dkgreen}{rgb}{0,0.6,0}
\definecolor{gray}{rgb}{0.5,0.5,0.5}
\definecolor{mauve}{rgb}{0.58,0,0.82}

\lstset{frame=tb,
  language=Java,
  aboveskip=3mm,
  belowskip=3mm,
  showstringspaces=false,
  columns=flexible,
  basicstyle={\small\ttfamily},
  numbers=none,
  numberstyle=\tiny\color{gray},
  keywordstyle=\color{blue},
  commentstyle=\color{dkgreen},
  stringstyle=\color{mauve},
  breaklines=true,
  breakatwhitespace=true,
  tabsize=3
}
\UseRawInputEncoding
\date{\today}

\begin{document}
\maketitle

\begin{lstlisting}
  import java.io.*;
  import java.util.*;
  import ithakimodem.*;
  
  /* My virtual modem class */
  public class myVirtualModem {
  
      int speed = 50000;  // modem speed
      int timeOut = 240000; // 4 minutes per session
      private Modem modem; // my modem instance
  
      /* Main function where i call everything and save results */
      public static void main(String[] args) throws InterruptedException {
          myVirtualModem userApplication = new myVirtualModem();
  
          userApplication.echoPackage("E3181\r","/home/teras/IdeaProjects/Computer_Networks_I_Project/
          results/echo_package_time_results");
          userApplication.Image_request("M1902\r","/home/teras/IdeaProjects/Computer_Networks_I_Project/
          results/errorFreeImage.png" );
          userApplication.Image_request("G3585\r","/home/teras/IdeaProjects/Computer_Networks_I_Project/
          results/errorImage.png" );
          userApplication.arqRequest("Q6322\r", "R9512\r","/home/teras/IdeaProjects/Computer_Networks_I_Project/
          results/Arq.txt","/home/teras/IdeaProjects/Computer_Networks_I_Project/
          results/nackResults.txt" );
          userApplication.Image_request("P5191T=225735403737T=225740403735T=225742403733T=225735403733\r", "/home/teras/IdeaProjects/Computer_Networks_I_Project/
          results/gpsImage.jpeg");
          userApplication.Gps_request("P5191R=1000135\r","/home/teras/IdeaProjects/Computer_Networks_I_Project/
          results/gpsPackages.txt" );
  
      }
  
      // Iniatilize my virtual modem every time i want to do something
      public void myVirtualModem() {
  
          int k;
          String rxmessage = ""; // buffer for writing the message from ithaki
  
          modem = new Modem();
          modem.setSpeed(speed);
          modem.setTimeout(timeOut);
          modem.open("ithaki");    // action to talk too ithaki
          modem.write("atd2310ithaki\r".getBytes());  // Connect local modem to remote ithaki modem.
  
          //System.out.println("end of welcome message");
          // Welcome message from ithaki when connected or error
          for (; ; ) {
              try {
                  k = modem.read();   // k 0-255, blocking function
                  if (k == -1) {  // something went wrong
                      System.out.println("error message");
                      break;
                  }
                  //System.out.print((char) k);
                  rxmessage = rxmessage + (char) k;
                  if (rxmessage.indexOf("\r\n\n\n") > -1) {          // delimeter for welcome message
                      System.out.println("end of welcome message");
                      break;
                  }
              } catch (Exception x) {
                  break;
              }
          }
          //modem.close();    // I dont close the modem here, i do it in the end of every function
      }
  
      // Function of echo package
      public int echoPackage(String pack_code, String path) {
          System.out.println("EchoPackage");
  
          // Init  virtual modem
          this.myVirtualModem();
          // IO to store results
          BufferedWriter bw = null;
          File file;
          FileWriter fw;
  
          try {
              // connection timeout for echo package
              long connectionStart = System.currentTimeMillis(); // Connection start here to stop when timeout time have passed
              long connectionFinish = connectionStart + timeOut; // 240000/1000 = 240 seconds = 4 minutes
              long packageTxTime = 0, packageRxTime = 0;  // transmit and receive times for a package
              int numOfPackages = 0;  // Number of packages received
              long avgTime = 0;  // Average time for packages
              String rxmessage = "";  // Buffer for storing each package
              int k; // buffer for each read
  
              // File and buffer IO to store results
              file = new File(path);
              fw = new FileWriter(file);
              bw = new BufferedWriter(fw);
  
              // While for capturing every package
              while ((System.currentTimeMillis() < connectionFinish)  && (numOfPackages < 9000)) {  // Timeout and number of packages check
                  rxmessage = "";
                  // Write echo package code
                  modem.write(pack_code.getBytes());
                  packageTxTime = System.currentTimeMillis(); // Save current time
                  for (; ; ) { // Read echo package
                      try {
                          k = modem.read();
                          //System.out.print((char) k);
                          \begin{verbatim} rxmessage = rxmessage + (char) k;
                          if (rxmessage.indexOf("PSTOP") > -1) { // Package delimeter
                              //System.out.println("package is here");
                              break;
                          }
                          if (k == -1) {
                              System.out.println("Maybe the packet is here\n");
                              break;
                          }
                      } catch (Exception x) {
                          System.out.println(x);
                          return 0;
                      }
                  }
                  packageRxTime = System.currentTimeMillis();  // Save the finish time of receiving a package
                  numOfPackages += 1; // Count the package received
                  avgTime = avgTime + (packageRxTime - packageTxTime);  // Add the time of every package
                  String time = String.valueOf(packageRxTime - packageTxTime); // save the time in a string
  
                  bw.write(time); // Write the time of the package
                  bw.newLine();  // add a newline
              }
              avgTime = avgTime / numOfPackages; // Calculate average time
              // Print average time, number of packages and total time
              System.out.println("avg time is " + avgTime);
              System.out.println("number of packages received " + numOfPackages);
              System.out.println("in time " + avgTime * numOfPackages);
  
              // Write them in the file, do not needed anynore
              //bw.newLine();
              //bw.write("avg time is " + avgTime + "\n");
              //bw.write("number of packages received " + numOfPackages + "\n");
              //bw.write("in time " + avgTime * numOfPackages + "\n");
  
              // Flush and close bufferwriter
              bw.flush();
              bw.close();
  
              // Close modem
              modem.close();
          } catch (Exception x) {
              System.out.println("\nException in echoPackage! ");
              return 0;
          }
          return 0;
      }
  
      // Image request function
      public void Image_request(String pack_code, String file_path) {
          System.out.println("Image Request");
          // Init  virtual modem
          this.myVirtualModem();
  
          try {
              // IO to store images
              File file = new File(file_path);
              OutputStream image = new FileOutputStream(file);
              // Write image package code
              modem.write(pack_code.getBytes());
              String rxmessage = ""; // buffer for writing the message from ithaki
              int k;
              boolean flag = false;
  
              long packageTxTime = 0, packageRxTime = 0;  // transmit and receive times for a package
              packageTxTime = System.currentTimeMillis();  // Save current time
  
              for (; ; ) {
                  try {
                      k = modem.read();   // k 0-255, blocking function
                      if (k == -1) break;
                      rxmessage = rxmessage + (char) k;
                      //System.out.println(k);
  
                      if (rxmessage.indexOf("ÿØ") > -1) {  // Image start delimeter
                          image.write(255);
                          image.write(216);
                          System.out.println("start reading of image");
                          rxmessage = "";
                          flag = true;
                      }
                      if (flag) {  // Check if we started getting the image
                          image.write(k);
                      }
                      if (rxmessage.indexOf("ÿÙ") > -1) {  // Image end delimeter
                          System.out.println("end of image");
                          image.write(k);
                          break;
                      }
                  } catch (Exception x) {
                      break;
                  }
              }
              packageRxTime = System.currentTimeMillis();  // Save time when the image is received
              System.out.println("Finished receiving the image after " + (packageRxTime - packageTxTime) / 1000 + "seconds!"); // Print how much time it took
  
              // Close bufferwriter and virtual modem
              image.close();
              modem.close();
          } catch (Exception x) {
              System.out.println("Exception in Image request");
          }
      }
  
      // Gps request function
      public int Gps_request(String gps_code, String path) {
          System.out.println("Gps reuest");
          // Init  virtual modem
          this.myVirtualModem();
          // Write image package code
          modem.write(gps_code.getBytes());
          // IO to store gps data
          BufferedWriter bw = null;
          File file;
          FileWriter fw;
          String rxmessage = ""; // buffer for writing the message from ithaki
          int k;
  
          try {
              // IO init
              file = new File(path);
              fw = new FileWriter(file);
              bw = new BufferedWriter(fw);
  
              // Reading of gps data
              for (; ; ) {
                  try {
                      k = modem.read();   // k 0-255, blocking entolh
                      if (k == -1) break;
                      //System.out.print((char) k);
                      rxmessage = rxmessage + (char) k;
                      if (rxmessage.indexOf("STOP ITHAKI GPS TRACKING") > -1) { // GPS delimeter
                          System.out.println("\nend of gps message");
                          break;
                      }
                  } catch (Exception x) {
                      break;
                  }
              }
              // Write and flush GPS data
              bw.write(rxmessage);
              bw.flush();
              bw.close();
              // Close modem
              modem.close();
          }catch (Exception e){
              System.out.println("\nException in echoPackage! ");
              return 0;
          }
          return 0;
      }
  
      // ARQ request function
      public void arqRequest(String ackCode, String nackCode, String pathArq, String pathResults) {
          System.out.println("arq Request");
          // Init  virtual modem
          this.myVirtualModem();
  
          // IO to store gps data
          BufferedWriter bwArq = null, bwResults = null;
          File fileArq, fileResults;
          FileWriter fwArq, fwResults;
  
          String rxmessage = ""; // buffer for writing the message from ithaki
          long connectionStart = System.currentTimeMillis();
          long connectionEnd = connectionStart + timeOut; // 240000/1000 = 240 seconds = 4 minutes
          long packageTxTime = 0, packageRxTime = 0, packageTime;  // transmit and receive times for a package
          int numOfAttemps = 0;
          boolean correctTransmit = true; // If the FCS == XOR result
          
          int k, fcs;
          String[] parseRxmessage; // Save parsed package to check if it is correct
          char[]  xxx16;   // 16 byte message form package
          byte x;  // to do the XOR check
          int numOfPackages = 0;   // NUmber of packages
  
          try {
              // File and buffer io
              fileArq = new File(pathArq);
              fileResults = new File(pathResults);
              fwArq = new FileWriter(fileArq);
              bwArq = new BufferedWriter(fwArq);
              fwResults = new FileWriter(fileResults);
              bwResults = new BufferedWriter(fwResults);
  
              while (System.currentTimeMillis() < connectionEnd) {
  
                  // If the previous package is correct then we ask for a new, else we write nack code to retransmit previous package
                  if(correctTransmit){
                      numOfAttemps = 1;  // Save how many attempts we did for one package
                      packageTxTime = System.currentTimeMillis();
                      modem.write(ackCode.getBytes());
                  }else {
                      numOfAttemps++;  // increase if we have a wrong transmit
                      modem.write(nackCode.getBytes());
                  }
                  rxmessage = "";  // buffer for writing the message from ithaki
                  for (;;) {
                      try {
                          k = modem.read();   // k 0-255, blocking function
                          //System.out.print((char) k);
                          rxmessage = rxmessage + (char) k;
  
                          if (rxmessage.indexOf("PSTOP") > -1 || k == -1) // Package delimeter
                              break;
  
                      } catch (Exception e) {
                          System.out.println("Catched exception e");
                          break;
                      }
                  }
                  //System.out.println(rxmessage);
                  packageRxTime = System.currentTimeMillis(); // Save time of package arrival
                  packageTime = packageRxTime - packageTxTime;  // Calculate time
                  parseRxmessage = rxmessage.split(" "); // Parse package based on spaces
  
                  fcs =  Integer.parseInt(parseRxmessage[5]); // FCS code is the 6th element
                  //System.out.println(fcs);
                  xxx16 = parseRxmessage[4].toCharArray();   // 16byte message is the 5th element
                  //System.out.println(xxx16);
                  x = (byte) xxx16[1]; // But the message is in <16xxx> so we take the 2 to 17 element
                  for(int i=2; i<xxx16.length-1; i++)
                      x = (byte) (x^xxx16[i]);  // XOR bytes of message
  
                  //System.out.println(x);
                  if((int)x == fcs){ // Check if FCS == XOR in decimal
                      // If correct increase packages, save true transmit and save package
                      numOfPackages++;
                      correctTransmit = true;
                      bwResults.write((packageTime + " " + numOfAttemps + "\n"));
                      bwResults.flush();
                      bwArq.write(rxmessage + "\n");
                      bwArq.flush();
                      //arrayOfAttempts[numOfAttemps]++;
                  }else{
                      // Else false transmit, save fals transmit
                      correctTransmit = false;
                      bwArq.write( rxmessage + "  wrong package: " + parseRxmessage[3] + ") " + "\n");
                      bwArq.flush();
                  }
              }
              // Close bufferwritters
              bwArq.close();
              bwResults.close();
  
              // Close virtual modem
              modem.close();
          } catch (Exception e) {
              System.out.println("\nException in ArqRequest! ");
          }
      }
  }
\end{lstlisting}
\end{document}
